%#!make uptex-manual.pdf
\documentclass[uplatex,a4paper,11pt,nomag]{jsarticle}
\usepackage[textwidth=42zw,lines=40,truedimen,centering]{geometry}

%%%%%%%%%%%%%%%%
% additional packages
\usepackage{amsmath}
\usepackage{array}\usepackage[all]{xy}
\SelectTips{cm}{}
%\usepackage[dvipdfmx]{graphicx}
\usepackage[T1]{fontenc}
\usepackage{booktabs,enumitem,multicol}
\usepackage[defaultsups]{newpxtext}
\usepackage[zerostyle=c,straightquotes]{newtxtt}
\usepackage{newpxmath}
\usepackage[dvipdfmx,hyperfootnotes=false]{hyperref}
\usepackage{pxjahyper}
\usepackage{hologo}
\usepackage{makeidx}\makeindex

% common
\usepackage{../ptex-manual/ptex-manual}

\def\code#1{\texttt{#1}}

%%%%%%%%%%%%%%%%
\makeatletter
\setlist{leftmargin=2zw}
\setlist[description]{labelwidth=2zw,labelindent=1zw,topsep=\medskipamount}

\def\>{\ifhmode\hskip\xkanjiskip\fi}

\def\tsp{_{\mbox{\fontsize\sf@size\z@\ttfamily \char32}}}
\def\tpar{_{\mbox{\fontsize\sf@size\z@\ttfamily \string\par}}}
\def\tign{_{\mbox{\fontsize\sf@size\z@\selectfont --}}}

\def\tableautorefname{表}
\def\figureautorefname{図}
\def\HyRef@autoref#1#2{%
  \begingroup
    \Hy@safe@activestrue
    \expandafter\HyRef@autosetref\csname r@#2\endcsname{#2}{#1}%
  \endgroup\textcompwordmark %欧文ゴースト
}
\makeatother

\usepackage{shortvrb}
\MakeShortVerb*{|}
%%%%%%%%%%%%%%%%

% logos
\def\epTeX{$\varepsilon$-\pTeX}\def\eTeX{$\varepsilon$-\TeX}
\def\eupTeX{$\varepsilon$-\upTeX}\def\upTeX{u\pTeX}
\def\pTeX{p\kern-.10em\TeX}

\def\_{\leavevmode\vrule width .45em height -.2ex depth .3ex\relax}

\frenchspacing
\begin{document}
\catcode`\<=13
\title{\emph{\upTeX マニュアル}}
\author{日本語\TeX 開発コミュニティ\null
\thanks{\url{https://texjp.org},\ e-mail: \texttt{issue(at)texjp.org}}}
\date{version p\the\ptexversion.\the\ptexminorversion\ptexrevision
             -u\the\uptexversion\uptexrevision,\today}
\maketitle

\tableofcontents

\newpage

%%%%%
\part{\upTeX の拡張と追加プリミティブ}

ここでは,\upTeX における\pTeX からの変更点と,追加または拡張された
プリミティブを説明する.

\section{\upTeX で利用可能な文字}
\label{sec:uptexchar}

\pTeX で直接使える文字集合はJIS~X~0208(JIS第1,2水準)に限られていたが,
内部コードがUnicode化された\upTeX ではUnicodeの文字を直接扱える.

[TODO] もう少し詳しく.起動時オプションの|uptex|も説明

\section{和文文字トークンと欧文文字トークンの区別}
%% https://github.com/texjporg/tex-jp-build/issues/43
\.{kcatcode}% pTeX から大きく変更されている,ハングルが19

\begin{cslist}
 \csitem[\.{enablecjktoken}]
  和文文字と欧文文字の区別について,\.{kcatcode}の設定に従う.
  \.{kcatcode}が15なら欧文扱い,16以上なら和文扱いにする.
  \upTeX のデフォルトはこの状態である.

 \csitem[\.{disablecjktoken}]
  全てのUnicode文字を欧文扱いにする(単なるバイト列として扱う).
  入力に関して8-bit欧文\TeX と互換になる.

 \csitem[\.{forcecjktoken}]
  ASCII文字以外のUnicode文字を(\.{kcatcode}にかかわらず)
  強制的に和文扱いにする.
  このとき,\.{kcatcode}が15の文字は「その他和文」(18)扱いになる.
\begin{verbatim}
\forcecjktoken\kcatcode`西=15
\message{\the\kcatcode`@ % ==> 18
\ifcat 西@ T\else F\fi}% ==> T
\end{verbatim}
\end{cslist}



\.{kchar}
\.{kchardef}

% \.{ucs}は元々\upTeX による拡張であったが,
% \pTeX~p3.10.0でコミュニティ版\pTeX にバックポートされた.

% [TODO] 補足として以下の pTeX <-> upTeX の違いもまとめる?
% https://github.com/texjporg/tex-jp-build/issues/36
% \.{kansujichar}
% \.{inhibitxspcode}

\section{バージョン番号}

\pTeX~p3.8.0に\.{ptexversion}, \.{ptexrevision}が実装されたのと同時に,
\upTeX でもバージョン番号を取得する以下のプリミティブが追加された.

\begin{cslist}
 \csitem[\.{uptexversion}, \.{uptexrevision}]
  \upTeX のバージョン番号はu$x{.}y$の形式となっており,
  それらを取得するための命令である.
  \.{uptexversion}は$x$の値を内部整数で返し,
  \.{uptexrevision}はその後ろの「${.}y$」を文字列で返す.
\end{cslist}


\newpage


\begin{thebibliography}{99}
\end{thebibliography}

\newpage
\printindex


\end{document}
